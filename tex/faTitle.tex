% !TeX root=../main.tex
% در این فایل، عنوان پایان‌نامه، مشخصات خود، متن تقدیمی‌، ستایش، سپاس‌گزاری و چکیده پایان‌نامه را به فارسی، وارد کنید.
% توجه داشته باشید که جدول حاوی مشخصات پروژه/پایان‌نامه/رساله و همچنین، مشخصات داخل آن، به طور خودکار، درج می‌شود.
%%%%%%%%%%%%%%%%%%%%%%%%%%%%%%%%%%%%
% دانشگاه خود را وارد کنید
\university{دانشگاه تهران}
% پردیس دانشگاهی خود را اگر نیاز است وارد کنید (مثال: فنی، علوم پایه، علوم انسانی و ...)
\college{پردیس دانشکده‌های فنی}
% دانشکده، آموزشکده و یا پژوهشکده  خود را وارد کنید
\faculty{دانشکدهٔ برق و کامپیوتر}
% گروه آموزشی خود را وارد کنید (در صورت نیاز)
\department{}
% رشته تحصیلی خود را وارد کنید
\subject{مهندسی برق}
% گرایش خود را وارد کنید
\field{مخابرات امن و رمزنگاری}
% عنوان پایان‌نامه را وارد کنید
\title{جلوگیری از تقلب برای احراز هویت مبتنی بر تشخیص چهره}
% نام استاد(ان) راهنما را وارد کنید
\firstsupervisor{دکتر محمد علی اخایی}
\firstsupervisorrank{استادیار}
%\secondsupervisor{دکتر راهنمای دوم}
%\secondsupervisorrank{استادیار}
% نام استاد(دان) مشاور را وارد کنید. چنانچه استاد مشاور ندارید، دستورات پایین را غیرفعال کنید.
%\firstadvisor{دکتر مشاور اول}
%\firstadvisorrank{استادیار}
%\secondadvisor{دکتر مشاور دوم}
% نام داوران داخلی و خارجی خود را وارد نمایید.
\internaljudge{دکتر داور داخلی}
\internaljudgerank{دانشیار}
\externaljudge{دکتر داور خارجی}
\externaljudgerank{دانشیار}
\externaljudgeuniversity{دانشگاه داور خارجی}
% نام نماینده کمیته تحصیلات تکمیلی در دانشکده \ گروه
\graduatedeputy{دکتر نماینده}
\graduatedeputyrank{دانشیار}
% نام دانشجو را وارد کنید
\name{میثم}
% نام خانوادگی دانشجو را وارد کنید
\surname{شهبازی دستجرده}
% شماره دانشجویی دانشجو را وارد کنید
\studentID{810197289}
% تاریخ پایان‌نامه را وارد کنید
\thesisdate{اردیبهشت 1401}
% به صورت پیش‌فرض برای پایان‌نامه‌های کارشناسی تا دکترا به ترتیب از عبارات «پروژه»، «پایان‌نامه» و «رساله» استفاده می‌شود؛ اگر  نمی‌پسندید هر عنوانی را که مایلید در دستور زیر قرار داده و آنرا از حالت توضیح خارج کنید.
%\projectLabel{پایان‌نامه}

% به صورت پیش‌فرض برای عناوین مقاطع تحصیلی کارشناسی تا دکترا به ترتیب از عبارت «کارشناسی»، «کارشناسی ارشد» و «دکتری» استفاده می‌شود؛ اگر نمی‌پسندید هر عنوانی را که مایلید در دستور زیر قرار داده و آنرا از حالت توضیح خارج کنید.
%\degree{}
%%%%%%%%%%%%%%%%%%%%%%%%%%%%%%%%%%%%%%%%%%%%%%%%%%%%
%% پایان‌نامه خود را تقدیم کنید! %%
\dedication
{
{\Large این اثر ناچیز تقدیم می‌شود به :}\\
\begin{flushleft}{
	\huge
	176 امید \\
	\vspace{7mm}
	و\\
	\vspace{7mm}
	آرزوی پرپر شده ...
}
\end{flushleft}
}
%% متن قدردانی %%
%% ترجیحا با توجه به ذوق و سلیقه خود متن قدردانی را تغییر دهید.
\acknowledgement{
این پایان‌نامه در زمان همه‌گیری ویروس کرونا، انجام شده است. در زمانی که محدودیت‌های کرونایی موجب غیرحضوری شدن آموزش‌های دانشگاهی شده است. در این شرایط دشوار، حمایت‌های بی‌دریغ جناب آقای دکتر محمدعلی اخایی، پیش از پیش به چشم آمد. بر خود لازم می‌دانم از ایشان به‌دلیل پی‌گیری‌های مرتب جهت پیشبرد پایان‌نامه در این شرایط کرونایی تشکر و قدردانی کنم.
همچنین از آقایان رامین طوسی و سید امین حبیبی به‌علت مشاوره و راهنمایی‌های ارزنده تشکر می‌کنم. همچنین از آقای پویا نریمانی به‌علت مساعدت در اتصال از راه دور به رایانه‌های موجود در آزمایشگاه مخابرات امن و رمزنگاری، تشکر می‌کنم
%از همکاری و مساعدت‌های دکتر ... مسئول تحصیلات تکمیلی و سایر کارکنان دانشکده بویژه سرکار خانم ... کمال تشکر را دارم.

و در پایان، بوسه می‌زنم بر دستان خداوندگاران مهر و مهربانی، پدر و مادر عزیزم و بعد از خدا، ستایش می‌کنم وجود مقدس‌شان را و تشکر می‌کنم از خانواده عزیزم به پاس عاطفه سرشار و گرمای امیدبخش وجودشان، که بهترین پشتیبان من بودند.
}
%%%%%%%%%%%%%%%%%%%%%%%%%%%%%%%%%%%%
%چکیده پایان‌نامه را وارد کنید
\fa-abstract{
یکی از روش‌های احراز هویت خودکار، استفاده از چهره کاربر است. با توجه به پیشرفت‌های چشم‌گیر در حوزه تشخیص چهره، استفاده از چهره محبویت خاصی پیدا کرده است. در عین حال، استفاده از چهره برای احراز هویت، روشی به‌طور کامل امن نیست و فرد مهاجم می‌تواند با استفاده از چاپ کردن چهره فرد هدف، یا بازپخش ویدیویی از او، به‌جای فرد هدف، احراز هویت انجام دهد. از این رو روش‌ها و الگوریتم‌هایی در این حوزه برای بهبود امنیت سیستم‌های احراز هویت با چهره، در تحقیقات دانشگاهی و صنعتی توسعه داده شده است. هدف از این پژوهش­ها تشخیص و تمییز تصویر چهره واقعی از تصویر چهره تقلبی ارائه شده توسط فرد مهاجم است. با رشد استفاده از روش‌های یادگیری عمیق در مسائل بینایی ماشین، در این حوزه نیز از الگوریتم‌های یادگیری عمیق برای طبقه‌بندی تصویر واقعی در مقابل تصاویر تقلبی ارائه شده توسط فرد مهاجم، استفاده شده است. در این پایان‌نامه با ترکیب روش کلاسیک بینایی ماشین و روش‌های یادگیری عمیق، یک عملگر جدید برای جایگزین کردن در یکی از لایه‌های کانولوشن ارائه شده است. همچنین برای افزایش دقت طبقه بندی بین دو دسته تصویر واقعی و تقلبی تابع هزینه­ای برای دسته­بندی دودویی با حاشیه ارائه شده است که افزودن این حاشیه باعث می­شود نمونه­های دو کلاس از یک­دیگر فاصله داشته باشند. علاوه بر این برای افزایش  قابلیت تعمیم‌پذیری شبکه، تابع هزینه­ی متریک اختصاصی برای مسئله کشف تقلب در چهره، با کمک گرفتن از شناسه اشخاص پیشنهاد شده است. همچنین نتایج روی برخی از دیتاست‌های معروف در این حوزه، گزارش شده و عملکرد کلی الگوریتم پیشنهادی به همراه سرعت اجرا بحث شده است.
}
% کلمات کلیدی پایان‌نامه را وارد کنید
\keywords{ احراز هویت، استفاده از چهره، امینت سیستم‌های احراز هویت، ترکیب روش‌های بینایی ماشین با یادگیری عمیق، تابع هزینه با حاشیه، بایومتریک، تابع هزینه متریک اختصاصی}
% انتهای وارد کردن فیلد‌ها
%%%%%%%%%%%%%%%%%%%%%%%%%%%%%%%%%%%%%%%%%%%%%%%%%%%%%%
