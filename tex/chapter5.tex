% !TeX root=../main.tex
\chapter{نتیجه‌گیری و کارهای آینده}
%\thispagestyle{empty} 
\section{نتیجه‌گیری}
در این پایان‌نامه به بررسی روش‌های موجود در حوزه امنیت سیستم‌های احراز هویت با استفاده از چهره پرداخته شد. روش‌های موجود به‌صورت عمده از سیگنال‌های کمکی نظیر عمق استفاده کرده‌اند. همچنین در بسیاری از روش‌ها از فریم‌های متوالی ویدئو برای استنتاج در مورد زنده یا تقلبی بودن چهره استفاده شده است. در این پایان‌نامه روشی مبتنی بر استفاده از تنها یک فریم توسعه داده شده است. همچنین روش پیشنهادی نیازی به عمق به‌عنوان سیگنال کمکی ندارد. با‌این‌وجود روش پیشنهادی در پروتکل‌های اول و دوم در دو دیتاست بزرگ و جدید در این حوزه به دقت‌های رقابتی با روش‌های دیگر رسیده است.

از‌آنجا‌که قسمت اصلی پردازش در روش پیشنهادی بر پایه شبکه
 \lr{EfficientNet B0} 
است حجم محاسباتی روش پیشنهادی بهینه است. از‌نظر زمان پاسخ، به دلیل استفاده از یک فریم، سریع است.
در این پایان‌نامه عملگری جدید بر پایه \lr{LBP} پیشنهاد شده است که خاصیت آموزش پذیری شبکه‌های \lr{CNN} را دارد. همچنین به علت توسعه تابع هزینه با حاشیه، قابلیت تفکیک‌پذیری شبکه بیشتر شده است و استفاده از تابع هزینه مبتنی بر شناسه اشخاص موجب افزایش تعمیم‌پذیری شبکه شده است. مزیت استفاده از تابع هزینه در این است که افزایش دقت بدون افزودن بار محاسباتی به شبکه حاصل می‌شود. لذا در روش پیشنهادی با‌وجود آنکه زمان آموزش بیشتری نیاز دارد اما زمان ارزیابی و استفاده از شبکه تغییری نمی‌کند.
\section{پیشنهاد کارهای آینده}
در این پژوهش از
 \lr{EfficientNet B0} 
استفاده شده است. پژوهش‌های بعدی می‌تواند شامل استفاده از ساختار از‌ابتدا طراحی‌شده باشد. همچنین به‌منظور افزایش دقت استفاده از ساختار توجه
\LTRfootnote{Attention}
 در شبکه می‌تواند مفید باشد. استفاده از دنباله ویدیویی به‌جای یک فریم با یک ساختار جدید می‌تواند به افزایش دقت کمک کند. به‌منظور آنالیز بهتر بافت در تصویر، عملگر \lr{LBP} می‌تواند توسعه بیشتری داده شود به‌گونه‌ای که در تمامی لایه‌های شبکه به‌جای کانولوشن قرار بگیرد. همچنین تابع هزینه \lr{ARCB} می‌تواند مشابه روش
\cite{george2019deep}
روی یک صفحه مسطح به‌جای یک نورون نوشته شود. تابع هزینه مبتنی بر شناسه اشخاص می‌تواند به‌جای استفاده از شناسه اشخاص روی ویژگی‌های دیگر نظیر ابزار حمله بازنویسی شود. همچنین استفاده از عمق در کنار روش پیشنهادی ممکن است دقت بهتری به‌دست آورد.

در این پایان‌نامه تمرکز روی حملات چاپ و بازپخش بوده است. در این حوزه دیتاست‌هایی وجود دارند که شامل حملات استفاده از ماسک هستند. استفاده از روشی مشابه روش پیشنهادی روی دیتاست‌هایی که دارای تصاویر \lr{RGB} و \lr{IR} هستند نیز می‌تواند پژوهش بعدی باشد.

علاوه بر این، در این پایان‌نامه به‌منظور افزایش سرعت همگرایی، از روش بهینه‌سازی آدام و شبکه با وزن‌های آموزش‌دیده‌شده استفاده شده است. پژوهش بعدی می‌تواند شامل استفاده از بهینه سازی \lr{SGD} و شروع با وزن‌های تصادفی و آموزش روی تعداد ایپاک زیاد باشد که ممکن است نقطه بهینه بهتری را پیدا کند.
