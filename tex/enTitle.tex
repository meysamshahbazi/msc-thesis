% !TeX root=../main.tex
% در این فایل، عنوان پایان‌نامه، مشخصات خود و چکیده پایان‌نامه را به انگلیسی، وارد کنید.

%%%%%%%%%%%%%%%%%%%%%%%%%%%%%%%%%%%%
\latinuniversity{University of Tehran}
\latincollege{College of Engineering}
\latinfaculty{Faculty of Electrical and Computer Engineering}
\latindepartment{ Faculty of Electrical and Computer Engineering}
\latinsubject{Electrical Engineering}
\latinfield{Cryptography and Secure Communication}
\latintitle{Anti-spoofing for authentication based on face recognition}
\firstlatinsupervisor{Dr Mohammad Ali Akhaee}
%\secondlatinsupervisor{}
%\firstlatinadvisor{}
%\secondlatinadvisor{}
\latinname{Meysam}

\latinsurname{Shahbazi}
\latinthesisdate{May 2022}
\latinkeywords{Authentication, face use, security of authentication systems, combination of machine vision methods with deep learning, marginal cost function, biometric, proprietary metric cost function}

\en-abstract{
An automated authentication method that makes use of the user's face is one option. Because of substantial advancements in face recognition technology, facial recognition has become increasingly common. Face authentication is not totally safe, however, and an attacker can authenticate by printing the target person's face or replaying a video of him / her instead of the target person, which is a known vulnerability. Academic and industrial research have therefore developed methods and algorithms in this field to increase the security of face authentication systems, which have been tested and proven to work. The goal of this investigation is to determine the difference between the real face image and the phony face image supplied by the attacker. Deep learning algorithms have been used to classify the real image against the fake images provided by the attacker as a result of the increased use of deep learning methods in machine vision problems. Deep learning algorithms have been used to classify the real image against the fake images provided by the attacker. In this dissertation, a novel operator is presented to replace one of the convolution layers in a machine vision system by integrating the classical way of machine vision with deep learning methods. Additionally, in order to improve the classification accuracy between the two categories of real and counterfeit images, a cost function for binary classification with a margin has been proposed, which adds a margin to the samples of the two classes in order to space the samples of the two classes apart. In addition, in order to improve the network's scalability, a specific metric cost function for the problem of face fraud detection has been presented, which makes use of the identities of persons to do this. Furthermore, on certain well-known datasets in this sector, the results are presented, and the overall performance of the suggested approach is reviewed, as well as the execution speed of the algorithm under consideration.
}
